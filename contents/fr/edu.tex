\cvsection{cvcolor5}{LeckerliOne-Regular}{Éducation}

\cvsubsection{cvcolor5}{LeckerliOne-Regular}{Diplôme Canadien}
\cventry{Sept 2014 Juin 2016}{Maîtrise en informatique}{}{}{}{
	{\uqachref{Université du Québec à Chicoutimi (UQAC)}, Canada}
}

\begin{cvstate}
	\begin{cvtable}{2.5ex}
		{\tiny \ding{228}} & Gestion de Projet: \xphref{Étude de la méthode XP} et de la méthode Agile, Pratique d'une organisation SCRUM.\\
		{\tiny \ding{228}} & Projet de Recherche: \scphref{Metaheuristique Avancée}, \mpihref{Étude des frameworks OpenMP, MPI and OpenCL}.\\
		{\tiny \ding{228}} & Projets d'ingénieurie: Étude de AspectJ, \aeehref{Analyse des architectures, Proposition d'un projet centré sur le principle du \textit{mobile moment}}.\\
		{\tiny \ding{228}} & Conception de Moteur de Jeux, Initiation au \textit{hacking} et au \textit{foresics}.\\
	\end{cvtable}

	\begin{cvtable}{6.5em}
		\cvtag{cvcolor5!80}{projet étudiant} & \syfhref{\textbf{Shoot Your Fridge}: une App Android dédiée à la gestion de recettes de cuisines}
	\end{cvtable}
	\begin{cvtable}{2.5ex}
		{\tiny \ding{228}} & \designhref{Production des documents de conceptions} (Package, UseCases, Classes et Sequences).\\
		{\tiny \ding{228}} & Utilisation du téléphone comme un scanneur de codes barres.\\
		{\tiny \ding{228}} & Développement de l'interface graphique (XML) connectée au module Java qui récupère les recettes à partir d'un site de cuisine.
	\end{cvtable}

	\cvkeyline{cvcolor5!80}{projet étudiant}{\scphref{\textbf{Un solveur pour le problème des couvertures d'ensembles (SCP)}}}\\
	\begin{cvtable}{2.5ex}
		{\tiny \ding{228}} & Création d'algorithmes métaheuristiques à partir de l'analyse d'articles de recherche.\\
		{\tiny \ding{228}} & Développement de différents algorithmes génétiques dans le but de résoudre le SCP en \cplusplus.
	\end{cvtable}
\end{cvstate}

\cvsubsection{cvcolor5}{LeckerliOne-Regular}{diplôme français}
\cventry{Sept 2012 Juin 2016}{Diplôme d'Ingénieur}{}{}{}
	{\isimahref{Institut Supérieur d’Informatique, de Modélisation et de leurs Applications (ISIMA)}\\ à Clermont-Ferrand, France}

\begin{cvstate}
	\begin{cvtable}{2.5ex}
		{\tiny \ding{228}} & Spécialisation dans le Génie Logiciel et Systèmes Informatiques.\\
		{\tiny \ding{228}} & Projets de Recherche: Étude des programmes de simulation, Analyse des compileurs.\\
		{\tiny \ding{228}} & Développement avancé en \cplusplus, Java et \csharp, Étude d'outils avancés en Java et .NET\\
	\end{cvtable}

	\cvkeyline{cvcolor5!80}{projet étudiant}{\nomadhref{\textbf{Script d'automation dédié au programme scientifique NOMAD}}}\\
	\begin{cvtable}{2.5ex}
			{\tiny \ding{228}} & Développement d'un script en Python 3 qui automatise un programme d'optimisation complèxe développé par le \geradhref{GERAD}.
	\end{cvtable}
\end{cvstate}

\cventry{Sept 2010 Juin 2012}{Classes Préparatoires aux Grandes Ecoles}{}{Cours préparatoires pour les \emph{grandes écoles} françaises}{}
	{\cpgehref{CPGE Lycée Victor Grignard} à Cherbourg-Octeville, France}

\vspace{0.2cm}
