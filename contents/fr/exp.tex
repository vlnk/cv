\section{Experiences/Projets}

\cventry{2015}{Projet personnel}{\href{http://vlnk.github.io/}{Développement d'un site personnel}}{}{}{}

\begin{tabular}{@{\hspace{5.5em}}p{2.5ex}p{37em}}
	{\tiny \ding{228}} & Développement d'une vitrine pour mes différents travaux \\
	{\tiny \ding{228}} & Technologies : HTML5, Liquid templates, SCSS, JavaScript et Ruby \\
\end{tabular}

\vspace{0.2cm}

\cventry{Sept 2014 Avril 2015}{Projet en équipe}{\href{https://github.com/vlnk/ShootYourFridge}{Shoot Your Fridge}}{}{}{
    \textsc{UQAC: cours d'UML/POO Avancé, Chicoutimi, Québec
}}
\begin{tabular}{@{\hspace{5.5em}}p{2.5ex}p{37em}}
	{\tiny \ding{228}} & Application Android de recettes générées à partir de codes barres scannés.\\
	{\tiny \ding{228}} & Réalisation de la GUI (Android), récupération des recettes via HTTP.\\
	{\tiny \ding{228}} & Rédaction des documents de conceptions (Package, UseCases, Classes et Séquences) \\
\end{tabular}

\vspace{0.2cm}

\cventry{Avril 2014 Août 2014}{Stage Ingénieur}{Développement d'une visionneuse de données}{}{}{
	\textsc{\href{https://www.infineon.com/}{Infineon Technologies AG}, Regensburg, Allemagne}
}
\begin{tabular}{@{\hspace{5.5em}}p{2.5ex}p{37em}}
	{\tiny \ding{228}} & Création \textit{from scratch} d'un outils de visualisation de bases de données formattées en EXF (dévivé du XML) utilisée pour les travaux de recherches d'\textit{Infineon}.\\
	{\tiny \ding{228}} & Utilisation des outils .NET (C\# et XAML) et d'un framework applicatif de l'entreprise.\\
\end{tabular}
