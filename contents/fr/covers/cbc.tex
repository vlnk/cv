\recipient{CBC - Radio-Canada}{Maison de Radio-Canada, Avenue Viger Est\\
Montreal city, Quebec, H2L 3B5}
\date{\today}
\opening{Madame, Monsieur,}
\closing{Dans l'attente de votre réponse pour un entretien, je vous prie d'agréer, Madame, Monsieur, mes sincères salutations.}
\enclosure[Ci-joint]{curriculum vit\ae{}}

\makelettertitle

\textbf{Candidature spontanée : Recherche de stage dans les TI}

\introduction{}
%%% PARTIE ENTREPRISE
Je postule en candidature spontanée à votre entreprise \textit{Matricis Informatique}, car je suis très intéressé par votre proposition de \textbf{Développeur d'applications} que j'ai trouvé sur \textit{Stackoverflow Careers}.
%%%

Présentement, mes 3 années d'étude m'ont permis de travailler sur de nombreux projets d'envergures diverses dans de nombreux langages notamment en \textbf{C++, Python, C\# et Java}. Mon cours de \textbf{Services Web} m'a permis de découvrir les possibilités offertes par le framework .NET grâce à des outils tels que \textit{Linq} et \textit{ASP.NET}. Par ailleurs, lors de mon stage de deuxième année j'ai travaillé sur un projet d'envergure en Allemagne durant 5 mois entièrement \textbf{en Anglais}. Ce projet consistait à \textbf{la réalisation d'une application de visualisation de données en C\# que nous avons entièrement codée du prototype à l'application finale pour le client.} Cette application devait être intégrée à une plateforme conçue par les ingénieurs d'\textit{Infineon Technologies AG} et je devais respecter leur guide de style pour développer mes composants avec \textit{WPF}. Ce projet m'a familiarisé avec l'utilisation des différents outils mis à disposition par \textit{Microsoft} et de la \textit{MSDN}.

De plus, durant ma deuxième année d'étude à Clermont-Ferrand puis au Canada, j'ai étudié le langage \textbf{Java} à tous les niveaux : nous avons étudié en profondeur les bases du langage puis nous avons étudié des outils plus avancés avec les cours de \textit{Java Avancé} et de \textit{Programation Objet Avancé}. Par exemple, nous avons étudié le \textbf{multithreading}, les \textbf{interfaces utilisateur}, la \textbf{réflexivité} ou encore les outils d'ingénieure tels que \textbf{Ant} ou \textbf{Jenkins}. À cela s'ajoute mon expérience en \textbf{C++} que j'ai renforcé en codant deux programmes pour la recherche opérationnelle en résolvant deux problèmes : le voyageur de commerce et le problème de couverture d'ensembles.

J'ai travaillé en autodidacte sur mon site internet où j'ai exercé les langages JavaScript, HTML et CSS. Ce développement m'a permis de pratiquer mes connaissances dans ce langage et surtout d'améliorer mon statut de développeur web. En plus de ce travail, j'ai travaillé sur deux projets basés sur le langage \textbf{Python}. Le premier consistait à écrire un programme d'automatisation pour la recherche opérationnelle et analysait des programmes afin de paramétrer au mieux le logiciel \textsc{nomad} développé par des chercheurs de Montréal. En deuxième lieu, j'ai développé un programme de simulation de conversation \textsc{irc} pour un projet de web série intitulée \textit{Je suis toujours vivant}. Ce script utilise \textit{PyQt} pour faciliter son utilisation. \conclusion{}

\makeletterclosing
