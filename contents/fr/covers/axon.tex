\recipient{Axon-ID}{410, St-Nicolas, Suite 101\\
Montreal, Quebec, Canada, H2Y 2P5}
\date{\today}
\opening{Madame, Monsieur,}
\closing{Dans l'attente de votre réponse pour un entretien, je vous prie d'agréer, Madame, Monsieur, mes sincères salutations.}
\enclosure[Ci-joint]{curriculum vit\ae{}}

\makelettertitle

\textbf{Candidature spontanée : Stage de fin d'étude (Développeur Java)}

\introduction{}
%%% PARTIE ENTREPRISE
Je postule en candidature spontanée à votre entreprise \textbf{Axon} car je suis très intéressé par votre opportunité de \textit{Développeur Java} découverte sur \textit{LinkedIn}.

Durant mon année d'étude à Chicoutimi, j'ai assisté à un cours de \textit{Programmation Object Avancé} où nous avons étudié en détail les principes du langage \textbf{Java}. Ce cours s'est précisément concentré sur la qualité du code et l'utilisation des outils fournis par le langage pour fournir des programmes lisibles et modulables. Nous avons eu de plus un cours d'\textit{Architecture d'Application en Entreprise} où nous avons étudié \textbf{Spring}, \textbf{EJB} et quelques composants de \textbf{J2EE}.
%%%

Présentement, mes 3 années d'étude m'ont permis de travailler sur de nombreux projets d'envergures diverses dans de nombreux langages notamment en \textbf{C++, Python et C\#}. Mon cours de \textbf{Services Web} m'a permis de découvrir les possibilités offertes par le framework .NET grâce à des outils tels que \textit{Linq} et \textit{ASP.NET}. Par ailleurs, lors de mon stage de deuxième année j'ai travaillé sur un projet d'envergure en Allemagne durant 5 mois entièrement \textbf{en Anglais}. Ce projet consistait à \textbf{la réalisation d'une application de visualisation de données en C\# que nous avons entièrement codée du prototype à l'application finale pour le client.} Cette application devait être intégrée à une plateforme conçue par les ingénieurs d'\textit{Infineon Technologies AG} et je devais respecter leur guide de style pour développer mes composants avec \textit{WPF}. Ce projet m'a familiarisé avec l'utilisation des différents outils mis à disposition par \textit{Microsoft} et de la \textit{MSDN}.

J'ai travaillé en autodidacte sur mon site internet où j'ai exercé les langages JavaScript, HTML et CSS. Ce développement m'a permis de pratiquer mes connaissances dans ce langage et surtout d'améliorer mon statut de développeur web. En plus de ce travail, j'ai travaillé sur deux projets basés sur le langage \textbf{Python}. Le premier consistait à écrire un programme d'automatisation pour la recherche opérationnelle et analysait des programmes afin de paramétrer au mieux le logiciel \textsc{nomad} développé par des chercheurs de Montréal. En deuxième lieu, j'ai développé un programme de simulation de conversation \textsc{irc} pour un projet de web série intitulée \textit{Je suis toujours vivant}. Ce script utilise \textit{PyQt} pour faciliter son utilisation. \conclusion{}

\makeletterclosing
