\recipient{Averna}{87, rue Prince - Bureau 510\\
Montreal, Quebec, H3C 2M7\\
Canada}
\date{\today}
\opening{Madame, Monsieur,}
\closing{Dans l'attente de votre réponse pour un entretien, je vous prie d'agréer, Madame, Monsieur, mes sincères salutations.}
\enclosure[Ci-joint]{curriculum vit\ae{}}

\makelettertitle

Double diplômant de l'école d'ingénieur française ISIMA et de l'université UQAC, je recherche un stage de fin d'études afin d'appliquer mes connaissances en entreprise.
Je postule à votre offre de Développeur C++ au sein de votre entreprise \textit{Averna} car je suis très intéressé par vos produits. En effet, votre entreprise a développé une gamme impressionnante d'outils de tests essentiels dans de nombreux domaines : la défense, les transports, l'électronique, la recherche médicale et les réseaux de télécommunication. Cette diversité des solutions apportées pour l'automatisation des tests et l'élaboration de solutions efficaces encadrant ces tests reflète une maîtrise de ce processus exigeant, mais primordial dans le secteur informatique.

Présentement, mes 3 années d'étude m'ont permis de travailler sur de nombreux projets d'envergures diverses dans de nombreux langages notamment en \textbf{C++, Python et C\#}. Mes cours d'\textbf{Optimisation} et de \textbf{Métaheuristiques} m'ont sensibilisé à l'écriture d'algorithmes complexes en C++ afin de résoudre des problèmes théoriques en recherche opérationnelle. Ainsi, j'ai développé en deuxième année deux programmes permettant de résoudre certains problèmes de tournées de véhicules et ma troisième année d'étude m'a permis de développer un programme complet traitant le problème de couverture d'ensemble grâce à des articles de recherche préalablement choisis. Ces différents outils m'ont permis de me familiariser avec ce langage notamment avec la STL qui est un outil capital pour concevoir un code clair, concis et efficace. De plus, j'apprécie énormément les concepts liés à ce langage et je suis régulièrement les dernières avancées le concernant.

En outre, j'ai développé une simulation multiagents en C++ permettant de me familiariser avec les différents patrons de conception logicielle. J'ai approfondi l'étude de ces patrons lors de ma troisième année grâce à mes cours de \textbf{Programmation Objet Avancé} et d'\textbf{Architecture des Applications en Entreprise}. Ce dernier m'a permis de saisir l'importance de la conception architecturale et m'a aidé à approfondir mes connaissances relatives à l'UML. J'ai appris dans ces cours de nombreux outils importants dans la conception d'applications flexibles et facilement maintenables. De fait, nous avons étudié par exemple l'\textit{AspectJ}, le style architectural \textit{SOA} et les méthodes de \textit{Business Process Management}. J'ai appris à utiliser l'\textbf{UML} au cours de ces deux dernières années et j'ai réalisé avec deux collègues les documents de conceptions complets pour une application Android.

Par ailleurs, lors de mon stage de deuxième année j'ai travaillé sur un projet d'envergure en Allemagne durant 5 mois entièrement \textbf{en Anglais}. Ce projet consistait à \textbf{la réalisation d'une application de visualisation de données en C\# que nous avons entièrement codée du prototype à l'application finale pour le client.} Cette application devait être intégrée à une plateforme conçue par les ingénieurs d'\textit{Infineon Technologies AG} et je devais respecter leur guide de style pour développer mes composants avec \textit{WPF}. Ce projet m'a familiarisé avec l'élaboration d'interfaces graphiques et la séparation primordiale entre la partie \textit{design} et la partie \textit{logique} d'une application. J'ai complété cette expérience par la programmation d'un simulateur de conversation IRC. Ce programme a été codé en C++ et en Python avec l'utilisation de PyQt pour la GUI. Ces différentes expériences m'ont amené à approfondir mes compétences en tant que programmeur C++ et je souhaite partager celles-ci avec vos équipes.

\makeletterclosing
