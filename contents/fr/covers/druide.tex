\recipient{Druide informatique inc.}{1435 rue Saint-Alexandre, bureau 1040\\
Montreal, Quebec, H3A 2G4\\
Canada}
\date{\today}
\opening{Madame, Monsieur,}
\closing{Dans l'attente de votre réponse pour un entretien, je vous prie d'agréer, Madame, Monsieur, mes sincères salutations.}
\enclosure[Ci-joint]{curriculum vit\ae{}}

\makelettertitle

\introduction{}
Je postule à votre offre d'\textbf{Informaticien en T.A.L.} car je suis très intéressé pour travailler avec votre équipe pour l'aider aux développements de vos produits. J'utilise votre logiciel \textit{Antidote} tous les jours et je suis très enthousiaste à l'idée de contribuer au développement de vos différents projets.

Présentement, mes 3 années d'étude m'ont permis de travailler sur de nombreux projets d'envergures diverses dans de nombreux langages notamment en \textbf{C++, Python et C\#}. Mes cours d'\textbf{Optimisation} et de \textbf{Métaheuristiques} m'ont sensibilisé à l'écriture d'algorithmes complexes en C++ afin de résoudre des problèmes théoriques en recherche opérationnelle. Ainsi, j'ai développé en deuxième année deux programmes permettant de résoudre certains problèmes de tournées de véhicules et ma troisième année d'étude m'a permis de développer un programme complet traitant le problème de couverture d'ensemble grâce à des articles de recherche préalablement choisis. Ces différents outils m'ont permis de me familiariser avec ce langage notamment avec la STL qui est un outil capital pour concevoir un code clair, concis et efficace. De plus, j'apprécie énormément les concepts liés à ce langage et je suis régulièrement les dernières avancées le concernant.

En outre, j'ai approfondi mes connaissances en JAVA lors de ma troisième année grâce à mes cours de \textbf{Programmation Objet Avancé}. Nous y avons appris de nombreuses techniques avancées qui j'ai pu pratiquer sur mes comptes \href{https://github.com/vlnk/ShootYourFridge}{\textit{GitHub}} et \href{https://bitbucket.org/vlnk/tronpoa}{\textit{Bitbucket}}. Par ailleurs, lors de mon stage de deuxième année j'ai travaillé sur un projet d'envergure en Allemagne durant 5 mois entièrement \textbf{en Anglais}. Ce projet consistait à \textbf{la réalisation d'une application de visualisation de données en C\# que nous avons entièrement codée du prototype à l'application finale pour le client.} Cette application devait être intégrée à une plateforme conçue par les ingénieurs d'\textit{Infineon Technologies AG} et je devais respecter leur guide de style pour développer mes composants avec \textit{WPF}. Ce projet m'a familiarisé avec l'élaboration d'interfaces graphiques et la séparation primordiale entre la partie \textit{design} et la partie \textit{logique} d'une application. J'ai complété cette expérience par la programmation d'un simulateur de conversation IRC. Ce programme a été codé en C++ et en Python avec l'utilisation de PyQt pour la GUI. \conclusion{}

\makeletterclosing
