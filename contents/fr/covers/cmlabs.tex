\recipient{CM Labs}{645 Wellington, \#301\\
Montreal, Quebec, H3C 1T2\\
Canada}
\date{\today}
\opening{Madame, Monsieur,}
\closing{Dans l'attente de votre réponse pour un entretien, je vous prie d'agréer, Madame, Monsieur, mes sincères salutations.}
\enclosure[Ci-joint]{curriculum vit\ae{}}

\makelettertitle

Double diplômant de l'école d'ingénieur française ISIMA et de l'université UQAC, je recherche un stage de 6 mois afin d’appliquer mes connaissances en entreprise. Je postule à votre offre de stage de développement logiciel C++ afin de parfaire mes connaissances en C++ qui j'ai acquise durant mes 3 années d'étude d'ingénieur en France et au Canada. J'ai trouvé les travaux de \textsc{CM Labs} particulièrement intéressants puisqu'ils rejoignent certains de mes cours de \textit{Simulation} et de \textit{Métaheuristique} qui m'ont sensibilisé à l'importance des simulations en industrie dans le cadre de la gestion des risques et de l'optimisation des ressources.

Présentement, mes 3 années d'étude m'ont permis de travailler sur de nombreux projets d'envergures diverses dans de nombreux langages notamment en \textbf{C++ et en Python}. J'ai réalisé un programme C++11 afin de traiter le problème d'ensemble couvrant en métaheuristique ainsi qu'un simulateur de conversation IRC en C++14 et en Python (avec \textbf{PyQt}) pour la websérie \textit{Toujours Vivant}. J'aime m'informer sur les derniers progrès dans la conception du langage C++ et de la STL car j'ai un profond intérêt pour cette conception de l'informatique. En plus de ces projets, ma formation d'ingénieur est orienté dans \textbf{la conception d'architecture en UML} et \textbf{la gestion de la qualité} au sein des programmes informatiques. Dans le cadre de ces cours, nous avons été sensibilisés à la conception en amont des programmes via les modèles de conceptions et la rédaction des tests unitaires. De plus, nous avons pratiqué une organisation d'équipe SCRUM au travers de nombreux exercices qui ont abouti au prototypage d'une application Android permettant ainsi de trouver des recettes à partir de produits scannés par leur code-barre.

Par ailleurs, lors de mon stage de deuxième année j'ai travaillé sur un projet d'envergure en Allemagne durant 5 mois entièrement \textbf{en Anglais}. Ce projet consistait à la réalisation d'une application de visualisation de document en C\# que nous avons entièrement codé du prototype à l'application finale pour le client. Cette expérience m'a permis d'approfondir ma maîtrise de la Programmation Orientée Objet et je souhaite renouveler cette excellente expérience au sein de votre équipe.

\makeletterclosing
